\begin{DoxyAuthor}{Authors}
Vito Giovanni Castellana, \href{mailto:vitogiovanni.castellana@pnnl.gov}{\tt vitogiovanni.\-castellana@pnnl.\-gov} 

Marco Minutoli, \href{mailto:marco.minutoli@pnnl.gov}{\tt marco.\-minutoli@pnnl.\-gov} 
\end{DoxyAuthor}
\hypertarget{index_Introduction}{}\section{Introduction}\label{index_Introduction}
Emerging data analytics applications aim at processing unprecedented amount of data, posing novel challenges to the developers. One obviously desired feature when working with big data is scalability\-: intuitively, if the data does not fit in a single machine memory, then the application should be able to process data on multiple nodes of a cluster. When working at large scale, applications should also provide high-\/performance. In this context, since algorithms time and space complexity may grow exponentially with the size of the input data, performance does not necessarily mean fast, but may represent the difference between achievable or not. A typical approach to tackle these challenges is to highly customize a specific application implementation targeting a specific machine or architecture. An even more aggressive solution is based on hardware-\/software co-\/design which tailors not only the software implementation but also the actual target hardware, to solve a specific class of problems. Clearly, both of these approaches are characterized by significant development/design effort, high time-\/to-\/solution, and low portability/flexibility of both software and hardware. In order to address these issues, we propose S\-H\-A\-D\-: Scalable High Performance Algorithms and Data Structures for High Performance Data Analytics. S\-H\-A\-D is a software library, designed to achieve flexibility and portability, while providing scalability and high performance. Unlike other high performance data analytics frameworks, S\-H\-A\-D can support a variety of applications in several domains, including, but not limited to, graph processing, machine learning, and, data mining.\hypertarget{index_Goals}{}\section{Design Goals}\label{index_Goals}
S\-H\-A\-D is designed to offer several unique features\-: \begin{DoxyItemize}
\item {\bfseries flexibility}\-: S\-H\-A\-D components can be used to implement a variety of data analytics applications, without focusing on a specific domain. S\-H\-A\-D data structures support both applications that perform mostly read only operations, and, applications which perform frequent updates (e.\-g. streaming applications). \item {\bfseries scalability and performance}\-: S\-H\-A\-D provides scalable and efficient data structures, that can store, update and process T\-B-\/scale data. \item {\bfseries productivity}\-: S\-H\-A\-D data structures offer user friendly, S\-T\-L-\/like interfaces, improving developers productivity and possibly facilitating their adoption in already existing code bases. \item {\bfseries portability}\-: S\-H\-A\-D hides the low-\/level details of the underlaying architecture, so applications developed using S\-H\-A\-D data structures can run either on a single machine or on distributed systems.\end{DoxyItemize}
\hypertarget{index_Overview}{}\section{S\-H\-A\-D Design}\label{index_Overview}
S\-H\-A\-D is designed as a software stack, whose core components are\-: an Abstract Runtime A\-P\-I (A\-R A\-P\-I), a collection of general purpose algorithms and data-\/structures, and, domain specific libraries.\hypertarget{index_ARAPI}{}\subsection{Abstract Runtime A\-P\-I}\label{index_ARAPI}
The A\-R A\-P\-I expose a set of primitives for managing tasks execution and concurrency, and, data-\/movements on both clusters and single-\/node machines, with the purpose of hiding low level details of the underlying runtime systems and architectures. Also, this level of abstraction allow supporting multiple runtimes and architectures with relatively limited effort, consisting in mapping the A\-R A\-P\-I to the specific lower-\/level infrastructures.\hypertarget{index_algorithms}{}\subsection{General purpose algorithms and data structures}\label{index_algorithms}
On top of the A\-R A\-P\-I, we define the main S\-H\-A\-D data structures layer, which includes a multitude of general purpose data structures such as sets, maps, and, vectors. Such data-\/structures expose S\-T\-L-\/like interfaces, improving developers productivity and possibility facilitating their adoption in already existing code bases. S\-H\-A\-D data structures and algorithms are designed to store and process significant amount of data; given the nature of the problem, to achieve high performance it is required to support concurrent access to the data as well as concurrent data updates. Also, the S\-H\-A\-D library is flexible enough to allow exploring interesting features in novel H\-P\-D\-A applications, in particular streaming and reliability capabilities. For streaming, data-\/structures need to support dynamic allocations and updates, while still providing high-\/performance and scalability. For reliability, S\-H\-A\-D aims at tolerating faults, by preserving data-\/integrity, with limited time/space complexity and memory overhead.\hypertarget{index_extension}{}\subsection{S\-H\-A\-D extension}\label{index_extension}
The basic S\-H\-A\-D components themselves offer enough features to implement a wide variety of applications, and they represent the building blocks for S\-H\-A\-D extensions. Extensions are domain specific libraries, built on top of the S\-H\-A\-D main data structures. Examples of such libraries are a Table library, implementing for examople data structures and operations typical of relational databases (e.\-g. join, sort, etc), and a graph library, implementing graph data structures, access methods (e.\-g. for\-Each\-Edge, for\-Each\-Neighbor) and algorithms (e.\-g. community detection, page rank). Extensions can be developed as layered libraries, and can share and use multiple lower-\/level libraries. 